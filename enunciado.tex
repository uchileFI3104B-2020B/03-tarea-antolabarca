\documentclass[letter, 11pt]{article}
%% ================================
%% Packages =======================
\usepackage[utf8]{inputenc}      %%
\usepackage[T1]{fontenc}         %%
\usepackage{lmodern}             %%
\usepackage[spanish]{babel}      %%
\decimalpoint                    %%
\usepackage{fullpage}            %%
\usepackage{fancyhdr}            %%
\usepackage{graphicx}            %%
\usepackage{amsmath}             %%
\usepackage{color}               %%
\usepackage{mdframed}            %%
\usepackage[colorlinks]{hyperref}%%
%% ================================
%% ================================

%% ================================
%% Page size/borders config =======
\setlength{\oddsidemargin}{0in}  %%
\setlength{\evensidemargin}{0in} %%
\setlength{\marginparwidth}{0in} %%
\setlength{\marginparsep}{0in}   %%
\setlength{\voffset}{-0.5in}     %%
\setlength{\hoffset}{0in}        %%
\setlength{\topmargin}{0in}      %%
\setlength{\headheight}{54pt}    %%
\setlength{\headsep}{1em}        %%
\setlength{\textheight}{8.5in}   %%
\setlength{\footskip}{0.5in}     %%
%% ================================
%% ================================

%% =============================================================
%% Headers setup, environments, colors, etc.
%%
%% Header ------------------------------------------------------
\fancypagestyle{firstpage}
{
  \fancyhf{}
  \lhead{\includegraphics[height=4.5em]{LogoDFI.jpg}}
  \rhead{FI3104-1 \semestre\\
         Métodos Numéricos para la Ciencia e Ingeniería\\
         Prof.: \profesor}
  \fancyfoot[C]{\thepage}
}

\pagestyle{plain}
\fancyhf{}
\fancyfoot[C]{\thepage}
%% -------------------------------------------------------------
%% Environments -------------------------------------------------
\newmdenv[
  linecolor=gray,
  fontcolor=gray,
  linewidth=0.2em,
  topline=false,
  bottomline=false,
  rightline=false,
  skipabove=\topsep
  skipbelow=\topsep,
]{ayuda}
%% -------------------------------------------------------------
%% Colors ------------------------------------------------------
\definecolor{gray}{rgb}{0.5, 0.5, 0.5}
%% -------------------------------------------------------------
%% Aliases ------------------------------------------------------
\newcommand{\scipy}{\texttt{scipy}}
%% -------------------------------------------------------------
%% =============================================================
%% =============================================================================
%% CONFIGURACION DEL DOCUMENTO =================================================
%% Llenar con la información pertinente al curso y la tarea
%%
\newcommand{\tareanro}{3}
\newcommand{\fechaentrega}{7/11/2020 21:59 hrs}
\newcommand{\semestre}{2020B}
\newcommand{\profesor}{Valentino González}
%% =============================================================================
%% =============================================================================


\begin{document}
\thispagestyle{firstpage}

\begin{center}
  {\uppercase{\LARGE \bf Tarea \tareanro}}\\
  Fecha de entrega: \fechaentrega
\end{center}


%% =============================================================================
%% ENUNCIADO ===================================================================
\noindent{\large \bf Problema}

Para planetas que orbitan cerca del Sol, como Mercurio, el potencial
gravitacional se puede escribir como:

$$U(r) = - \dfrac{GM_\odot m}{r} + \alpha\dfrac{GM_\odot m}{r^2}$$

\noindent donde $G$ es la constante de gravitación universal, $M_\odot$ es la
masa del Sol, $m$ es la masa del planeta, $r$ es la distancia entre el planeta
y el Sol, y $\alpha$ es un número pequeño. Esta corrección a la ley de
gravitación de Newton se debe a los efectos de los otros planetas del sistema
solar (particularmente Júpiter) y a efectos relacionados con relatividad
general.

Bajo este potencial, las órbitas siguen siendo planas pero ya no son cerradas
sino que precesan, es decir, el afelio (punto más lejano de la órbita) gira
alrededor del Sol.

En este problema exploraremos este tipo de órbitas y compararemos distintos
métodos de integración de la ecuación diferencial. También utilizaremos
programación orientada al objeto como ejercicio práctico para familiarizarnos
con este paradigma de programación.

\begin{enumerate}

  \item El archivo llamado \texttt{codigos/planeta.py} contiene el esqueleto de
    la clase Planeta.  Ud. debe implementar los métodos de esa clase. Los
    docstrings explican en qué debe consistir cada método. Ud. tiene libertad
    de mejorar los docstrings, y agregar atributos y métodos a la clase según
    le parezca conveniente para resolver el problema descrito a continuación.

    El archivo llamado \texttt{codigos/solucion\_usando\_planeta.py} muestra cómo
    incluir la clase \texttt{Planeta} en un script separado. Ud. también puede
    resolver todo dentro del mismo archivo, en cuyo caso puede descartar
    \texttt{solucion\_usando\_planeta.py}.

  \item Parta por estudiar el caso $\alpha=0$ y considere las siguientes
    condiciones iniciales:
    \begin{flalign*}
      x_0 &= 10\\
      y_0 &= 0\\
      v_x &= 0\\
    \end{flalign*}

    Además, utilice unidades tales que $GM_\odot m = 1$ y escoja $v_y$ según le
    parezca (pero asegúrese de que la energía total sea negativa).

    Integre la ecuación de movimiento por aproximadamente 5 órbitas usando los
    métodos de RK, Verlet, y Beeman. Grafique las órbitas y la energía total
    del sistema como función del tiempo en los 3 casos. Comente los resultados.

  \item Ahora considere el caso $\alpha=10^{-2.XXX}$ (donde XXX son los 3
    últimos dígitos de su RUT, antes del dígito verificador). Integre la
    ecuación de movimiento usando el método de Beeman por al menos 30 órbitas.
    Determine la velocidad angular de precesión. ¿Cómo lo hizo? En particular,
    ¿cómo determinó la posición del afelio? Grafique la órbita y la energía
    como función del tiempo.

\end{enumerate}

\begin{ayuda}
  \small
  \noindent {\bf Comentarios.} 

  1. Esta tarea pide explícitamente que utilice OOP (Object Oriented
  Programming) para su desarrollo. Es un ejercicio útil para aprender esta
  técnica.

  2. En esta tarea debe implementar Verlet y Beeman pero puede utilizar RK4 (u
  otro RK de orden superior) de alguna librería o de alguna tarea anterior.
  Debe indicar en su informe la procedencia del código utilizado para RK.

\end{ayuda}

%% FIN ENUNCIADO ===============================================================
%% =============================================================================

\vspace{1em}
\noindent\textbf{Instrucciones Importantes.}
\begin{itemize}

\item Evaluaremos su uso correcto de \texttt{python}. Si define una función
  relativametne larga o con muchos parámetros, recuerde escribir el
  \emph{docstring} que describa los parámetros que recibe la función, el
  output, y el detalle de qué es lo que hace la función. Recuerde que
  generalmente es mejor usar varias funciones cortas (que hagan una sola cosa
  bien) que una muy larga (que lo haga todo).  Utilice nombres explicativos
  tanto para las funciones como para las variables de su código. El mejor
  nombre es aquel que permite entender qué hace la función sin tener que leer
  su implementación ni su \emph{docstring}.

\item Su código debe aprobar la guía sintáctica de estilo
  (\href{https://www.python.org/dev/peps/pep-0008/}{\texttt{PEP8}}). En
  \href{http://pep8online.com}{esta página} puede chequear si su código aprueba
  \texttt{PEP8}.

\item Utilice \texttt{git} durante el desarrollo de la tarea para mantener un
  historial de los cambios realizados. La siguiente
  \href{https://education.github.com/git-cheat-sheet-education.pdf}{cheat
    sheet} le puede ser útil. {\bf Revisaremos el uso apropiado de la
  herramienta y asignaremos una fracción del puntaje a este ítem.} Realice
  cambios pequeños y guarde su progreso (a través de \emph{commits})
  regularmente. No guarde código que no corre o compila (si lo hace por algún
  motivo deje un mensaje claro que lo indique). Escriba mensajes claros que
  permitan hacerse una idea de lo que se agregó y/o cambió de un
  \texttt{commit} al siguiente.

\item Para hacer un informe completo, además de los gráficos que se piden
  explícitamente, Ud. debe decidir qué es interesante y agregar las figuras
  correspondientes. No olvide anotar los ejes e incluir una \emph{caption} que
  describa el contenido de cada figura. Tampoco olvide las unidades asociadas a
  las cantidades mostradas en los diferentes gráficos.

\item La tarea se entrega subiendo su trabajo a github. Clone este repositorio
  (el que está en su propia cuenta privada y debiese tener una url como\\
  \texttt{https://github.com/uchileFI3104B-2020B/03-tarea-nombredeusuario}),
  trabaje en el código y en el informe y cuando haya terminado asegúrese de
  hacer un último \texttt{commit} y luego un \texttt{push} para subir todo su
  trabajo a github.

\item El informe debe ser entregado en formato \texttt{pdf}, este debe ser
  claro sin información de más ni de menos. \textbf{Esto es muy importante, no
  escriba de más, esto no mejorará su nota sino que al contrario}. La presente
  tarea probablemente no requiere informes de más de 4 páginas en total
  (dependiendo de cuántas figuras incluya; esto no es una regla estricta, sólo
  una referencia útil).  Asegúrese de utilizar figuras efectivas y tablas para
  resumir sus resultados.

\item \textbf{Revise su ortografía.}

 \item Repartición de puntaje: 45\% implementación y resolución del problema
   (más o menos independientemente de la calidad de su código); 40\% calidad
   del reporte entregado: demuestra comprensión del problema y su solución,
   claridad del lenguaje, calidad de las figuras utilizadas; 5\% aprueba o no
   \texttt{PEP8}; 10\% diseño del código: uso efectivo de nombres de variables
   y funciones, docstrings, \underline{uso de git}, etc.

\end{itemize}

\end{document}
